\documentclass{article}
\usepackage[utf8]{inputenc}
\usepackage[brazil]{babel}
\usepackage{hyperref}

\title{Atividade Curricular em Cultura e Extensão}
\author{Nathan Benedetto Proença}
\date{7 de março de 2016}

\begin{document}
\maketitle
\section{Supervisor}
    Escolho Renzo Gonzalo Gómez Diaz, treinador do MaratonIME como meu supervisor.
\section{Objetivo}
    O objetivo da atividade curricular é melhor meu nível como programador
competitivo. A principal ideia para tornar isso possível é tornar sistemático
meus treinos, a fim de evitar complicações e facilitar o estudo em si. Com mais
disciplina também posso melhor gerenciar minhas tarefas que não estejam
relacionadas com programação competitiva e assim ter exclusividade no tempo
investido nos treinos.
\section{Tarefas}
    Pretendo garantir 5 horas por semana de simulado, durante todo o semestre.
Como são 19 semanas letivas, isso vai totalizar 95 horas apenas em competições.
Para dar margem a possíveis imprevistos, irei me comprometer com 90h em provas.
Esse vai ser o grosso do treino, por três motivos:
\begin{enumerate}
    \item Fazer diversos simulados evita que eu limite os conteúdos que eu
treino a tópicos que eu já tenho destreza, sendo uma forma natural de conhecer
novos problemas, técnicas e aplicações de algoritmos.
    \item Simulados são uma forma simples e viável de sempre ter problemas a
serem resolvidos, o que evita inércia.
    \item São uma forma fácil de comprovar as horas de comprometimento. Não vejo
como, por exemplo, comprovar que houve tempo investido em estudo ou em resolução
de problemas, pois mesmo produzindo material, é altamente questionável quantas 
horas foram envolvidas naquela tarefa.
\end{enumerate}
\section{Acompanhamento}
    Irei atualizando nesta página \url{http://nathanpro.github.io/mac0214/}.
Lá irei mostrar os contests que participei, problemas avulsos resolvidos e
qualquer material que eu produza sobre o assunto. Para facilitar meu controle
e minha motivação, há um contador regressivo que soma as horas de cada contest
que eu participei, e irei mantê-lo atualizado. Ele irá chegar no zero quando
eu completar as 90h que eu mencionei acima.
\end{document}
